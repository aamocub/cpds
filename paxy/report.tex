\documentclass[a4paper, 10pt]{article}
\usepackage[utf8]{inputenc} % Change according your file encoding
\usepackage{graphicx}
\usepackage{url}

\usepackage{hyperref}
\hypersetup{
    linktocpage,
    pdfborderstyle={/S/U/W 1},
    pdfborder=0 0 1,
}

\usepackage[nameinlink, noabbrev]{cleveref}
\usepackage{booktabs}
\usepackage{caption}
\captionsetup{labelfont=sc,justification=centerlast}
\captionsetup[table]{skip=3pt}

\usepackage{microtype}

\usepackage{titlesec}
\titleformat*{\subsection}{\itshape}
\renewcommand\thesubsection{\roman{subsection})}

%opening
\title{Seminar Report: Paxy}
\author{ \textbf{Castilla Rodríguez, Víctor} \and \textbf{Chavez Quevedo, Luis Fernando} \and \textbf{Amorós Cubells, Alfonso}
}
\date{\normalsize\today{}}

\begin{document}

\maketitle

\begin{center}
  Upload your report in PDF format.

  Use this LaTeX template to format the report.

	A compressed file (.tar.gz) containing all your source code files must be submitted together with this report\footnote{Describe in the report any design decision required to understand your code (if any)}.
\end{center}

\section{Introduction}

\textit{Introduce in a couple of sentences the seminar and the main topic related to distributed systems it covers.}

\section{Experiments}

\textit{Provide evidence of the experiments you did (e.g., use screenshots) and discuss the results you got. In addition, you may provide figures or tables with experimental results of the system evaluation. For each seminar, we will provide you with some guidance on which kind of evaluation you should do.}

\section{Open questions}

\textit{Try to answer all the open questions in the documentation. When possible, do experiments to support your answers.}

\subsection{\!Try introducing different delays in the
\texttt{promise} messages, \texttt{vote} messages, or both sent by the acceptor. Q) Does the
algorithm still terminate? Does it require more rounds? How does the impact
of message delays depend on the value of the timeout at the proposer?}

\Cref{tab:delays} shows the effect of varying message delay time.
If kept below timeout (\(< 2000\)\,ms), increasing delay only increases execution time, but does not affect the number of rounds
% (which remain at 1)
or messages.
However, as soon as this delay surpasses 2000\,ms (e.g.~by setting it at 2100\,ms), the number of rounds increases and execution is significantly slower, although the algorithm still terminates.

\begin{table}[htp]
\centering
\caption{Effect of message delays on rounds and execution time.}
\label{tab:delays}
\begin{tabular}{rrrr}
\toprule
Delay (ms)\! & No.~rounds & No.~messages & Time (ms)\! \\ \midrule
0            & 1          & 44           & 2018        \\
500          & 1          & 44           & 3052        \\
1000         & 1          & 44           & 4068        \\
2000         & 1          & 44           & 5168        \\
2100         & 2          & 51           & 8170        \\
3000         & 3          & 72           & 10935       \\ \bottomrule
\end{tabular}
\end{table}

\subsection{Avoid sending \texttt{sorry} messages by commenting the corresponding sentences in the acceptor. Q) Could you come to an agreement when \texttt{sorry} messages are not sent?}

\subsection{\!Try randomly dropping \texttt{promise} messages, \texttt{vote} messages, or both in the acceptor. If you drop too many messages a quorum will of course never be found, but we could probably lose quite many. Q) How does the drop ratio affect the number of rounds required to reach consensus? What percentage of messages can we drop until consensus is no longer possible?}

\Cref{tab:drop-chance} shows the correlation between the drop chance and the number of rounds it takes to terminate.
If the drop chance is \(\leq 30\%\), it usually only takes 2 rounds to reach consensus.
After that, the number of rounds increases significantly, needing more than 10 rounds if half of the packages are dropped.
For drop chances \(\geq 70\%\), the algorithm is unlikely to terminate.

\begin{table}[htp]
\centering
\caption{Average number of rounds depending on drop probability.}
\label{tab:drop-chance}
\begin{tabular}{rr}
\toprule
Drop chance\! & No.~rounds   \\ \midrule
0\%           & 1                 \\
10\%          & 2                 \\
20\%          & 2                 \\
30\%          & 2                 \\
40\%          & 5                 \\
50\%          & 13                \\
60\%          & 12                \\
70\%          & 16                \\
80\%          & DNF               \\ \bottomrule
\end{tabular}
\end{table}

\subsection{\!Try increasing the number of acceptors and proposers. Q) What is the impact (in the number of rounds needed to get consensus) of having more acceptors while keeping the same number of proposers? What if we have more proposers while keeping the same number of acceptors?}

We have already seen that only 1 round is needed with 3 proposers and 5 acceptors.
If we add one more acceptor, then it takes 2 rounds to reach consensus.
After that, increasing the number of acceptors (e.g.~to 20) does not increase the number of rounds.

On the other hand, altering the number of proposers has more impact on the number of rounds.
With 5 proposers and 5 acceptors, 3 rounds are needed; and with 10 proposers, the number of rounds reaches 5.
Varying the initial delays generally does not reduce the number of rounds.

\subsection{Adapt the \texttt{paxy} module to create the proposers in a remote Erlang instance (named paxy-pro) and to ensure that they can connect correctly to the
acceptors, which must be created in a different remote Erlang instance (named
paxy-acc).}
% Note that acceptors and proposers should not be modified, the acceptors have to use locally registered names, there should be a single `start' (and `stop') function that creates (and stops) all the processes, and the proposers must be created once the acceptors are up and running. Check the slides about Erlang to refresh how processes are created remotely, how names registered in remote nodes are referred, and how Erlang runtime should be started to run distributed programs.

\section{Personal opinion}

\textit{Provide your personal opinion of the seminar, indicating whether it should be included in next year's course or not.}

\end{document}
